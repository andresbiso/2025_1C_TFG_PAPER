% Esta sección es conocida como "Motivación" o "Problemática".

En el mundo actual, las personas se enfrentan a una escasez de tiempo más que
nunca. La aparición de diversas preocupaciones ha limitado la posibilidad de un
enfoque a largo plazo. En esta situación, la mejor manera de mejorar el sistema
educativo es utilizar un método que proporcione la mayor cantidad de información
en el menor tiempo posible.
\cite{article:microlearning_today_students_nikkhoo}

Los cambios tecnológicos, económicos y sociales actuales impulsan la necesidad
de nuevos conceptos y estrategias para apoyar el aprendizaje permanente. La
educación, incluida la formación en el trabajo, necesita transformaciones que
requieren renovación y formas innovadoras de adaptarse adecuadamente a nuestra
forma de vivir, trabajar y aprender hoy.
\cite{article:microlearning_buchem}

Los términos que describen cómo trabajamos y aprendemos hoy, como ''trabajadores
del conocimiento'' (''knowledge workers''), ''nativos digitales'' (''Digital
Natives''), ''inmigrantes digitales'' ("Digital Immigrants"), ''estudiantes del
nuevo milenio'' (''new millennium learners'') y la ''Generación Net'' (''Net
Generation''), reflejan algunos cambios esenciales en las sociedades modernas,
''donde las tecnologías digitales forman parte inseparable de la vida cotidiana''
(''where digital technologies form an inextricable part of daily life''). El
estilo de vida de la nueva generación, es decir, de los nacidos a partir de la
década de 1980, pero también de las generaciones anteriores, se está viendo
fuertemente influenciado por Internet y las tecnologías relacionadas. Tanto los
estudiantes jóvenes como los mayores llevan sus portátiles a clases o reuniones
de trabajo, usan teléfonos móviles e internet para fomentar las redes sociales,
emplean dispositivos digitales para jugar y crear contenido, o realizan
múltiples tareas simultáneamente.
\cite{article:microlearning_buchem}

Estas nuevas tecnologías digitales que permiten la creación de contenido
generado por el usuario han dado lugar a una tendencia hacia los microformatos
(''microformats''), es decir, información breve, sencilla y específica. Junto con
los sistemas de publicación personal, como blogs o wikis, se ha vuelto bastante
fácil para cualquiera crear su propio contenido, incluido el microcontenido
(''microcontent''). El microcontenido, es decir, ''información publicada en formato
breve'', se relaciona más con un "enfoque formal de cómo presentar" el contenido
que con la calidad inherente del contenido en sí. Algunos ejemplos de
microcontenido son los podcasts, las entradas de blog, las páginas wiki o los
mensajes cortos en Facebook o X. La creación, publicación y compartición de
microcontenido en la web abre nuevas posibilidades para formas de aprendizaje
implícitas, informales e incidentales, como el microaprendizaje
(''microlearning''), término que se refiere a actividades breves de aprendizaje
con microcontenido.
\cite{article:microlearning_buchem}