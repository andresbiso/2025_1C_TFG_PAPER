Se desarrollará un Learning Management System (LMS) con un enfoque
centrado en el microaprendizaje, permitiendo una capacitación más
flexible y adaptativa para los usuarios.  

\subsection{Arquitectura}  
La solución adoptará una arquitectura cliente-servidor, con el
propósito de garantizar un funcionamiento eficiente y escalable. El diseño de la
arquitectura toma como referencia los enfoques planteados en diversos estudios
previos, incluyendo:  

\begin{itemize}  
    \item \textit{Design and Implementation of E-Learning Management System
    using Service Oriented Architecture} \cite{article:design_lms_soa_jabr}  
    \item \textit{An implementable architecture of an e-learning system}
    \cite{inproceedings:elearning_architecture_liu}  
    \item \textit{Microservice architecture in E-learning}
    \cite{article:microservice_architecture_elearning_milovanovic}  
\end{itemize}  

Estos artículos han influenciado el diseño de la solución, proporcionando los
conecptos y definiciones que permiten crear los modelos estructurales que
permiten mejorar la eficiencia del sistema y optimizar la distribución de los
contenidos de microaprendizaje dentro de un entorno LMS.  

\subsection{Innovación}  
La principal característica diferenciadora de esta solución respecto a otras
disponibles en el mercado es su integración con plataformas de mensajería. Este
enfoque busca facilitar la notificación automática a los usuarios sobre nuevos
cursos disponibles, fomentando un ritmo de aprendizaje continuo y adaptable a
sus necesidades.  

Inicialmente, la integración se realizará con la plataforma Telegram,
permitiendo el envío de actualizaciones en tiempo real. No obstante, se
contempla la posibilidad de extender la compatibilidad a otras plataformas de
mensajería en futuras investigaciones, según la demanda y los requerimientos del
sistema.  

La decisión de implementar esta integración se fundamenta en los resultados
obtenidos en diversas investigaciones, tales como:  

\begin{itemize}  
    \item \textit{The Usability Of WhatsApp Messenger as Online
    Teaching-learning Media} \cite{article:whatsapp_online_teaching_darman}  
    \item \textit{Using WhatsApp to Support Communication in Teaching and
    Learning} \cite{inproceedings:whatsapp_support_teaching_ujakpa}  
    \item \textit{Perception of the use of LMS/i-Learn portal and Telegram}
    \cite{article:perceptions_lms_telegram_hassim}  
    \item \textit{Assessment of the Applicability of using Telegram as a
    Learning Management System} \cite{article:telegram_lms_elebaed}  
\end{itemize}  

Según el artículo \textit{Assessment of the Applicability of using Telegram as a
Learning Management System}:  

\begin{quote}  
Los resultados analizados de los experimentos diseñados concluyen que Telegram
es un medio aceptable para comunicar los materiales de aprendizaje (vídeos y
apuntes) y proporciona un nivel aceptable de interactividad en el aprendizaje.
Además, la tasa media de éxito de los estudiantes en los cursos impartidos a
través de Telegram fue mejor que la de los cursos impartidos en el aula.
\end{quote}  
\cite{article:telegram_lms_elebaed}

Con base en estos hallazgos, se ha decidido iniciar la integración con
Telegram como plataforma principal, asegurando una comunicación
efectiva entre el LMS y los usuarios, y mejorando la accesibilidad a los
contenidos educativos.  
