% Sección obligatoria.
% ¿Qué escribir en esta sección?
% Motivación (o problemática), se incluyen los antecedentes y una descripción técnica y detallada de la idea.
% Por último, se debe incluir una descripción del resto de las secciones del paper.
% Esta sección y marco teórico deben contener casi todas las referencias del paper.

% La problemática o motivación del paper es el inicio de la introducción y sirve para saber si el problema que plantea el paper es conocido,
% investigado y cuál es su impacto en el mundo real.
% En esa parte de la introducción es necesario incluir referencias para identificar la problemática.

% Luego de plantear la problemática, es necesario encuadrar las soluciones ya existentes para el problema así como la nueva tecnología que podría ayudar a
% solucionar el tema.
% En la introducción no se explica cómo funcionan los algoritmos, solamente se citan para entender la complejidad.

% La idea debe comenzar con: “En este trabajo, vamos a desarrollar…”, “El objetivo del paper es…”, “En este artículo se explican…"
% Escribir ¿De qué se trata su trabajo?

% La descripción de las secciones debe ser un parrafo con una descripción breve de las secciones.

% ¿Cómo debe escribirse esta sección?
% Abarca mínimo media página (idealmente debería finalizar casi al final de la página 1).

% \IEEEPARstart{E}{ste} paper desarrolla la siguiente idea.

\subsection{Problemática}
% Esta sección es conocida como "Motivación" o "Problemática".

En el mundo actual, las personas se enfrentan a una escasez de tiempo más que
nunca. La aparición de diversas preocupaciones ha limitado la posibilidad de un
enfoque a largo plazo. En esta situación, la mejor manera de mejorar el sistema
educativo es utilizar un método que proporcione la mayor cantidad de información
en el menor tiempo posible.
\cite{article:microlearning_today_students_nikkhoo}

Los cambios tecnológicos, económicos y sociales actuales impulsan la necesidad
de nuevos conceptos y estrategias para apoyar el aprendizaje permanente. La
educación, incluida la formación en el trabajo, necesita transformaciones que
requieren renovación y formas innovadoras de adaptarse adecuadamente a nuestra
forma de vivir, trabajar y aprender hoy.
\cite{article:microlearning_buchem}

Los términos que describen cómo trabajamos y aprendemos hoy, como ''trabajadores
del conocimiento'' (''knowledge workers''), ''nativos digitales'' (''Digital
Natives''), ''inmigrantes digitales'' ("Digital Immigrants"), ''estudiantes del
nuevo milenio'' (''new millennium learners'') y la ''Generación Net'' (''Net
Generation''), reflejan algunos cambios esenciales en las sociedades modernas,
''donde las tecnologías digitales forman parte inseparable de la vida cotidiana''
(''where digital technologies form an inextricable part of daily life''). El
estilo de vida de la nueva generación, es decir, de los nacidos a partir de la
década de 1980, pero también de las generaciones anteriores, se está viendo
fuertemente influenciado por Internet y las tecnologías relacionadas. Tanto los
estudiantes jóvenes como los mayores llevan sus portátiles a clases o reuniones
de trabajo, usan teléfonos móviles e internet para fomentar las redes sociales,
emplean dispositivos digitales para jugar y crear contenido, o realizan
múltiples tareas simultáneamente.
\cite{article:microlearning_buchem}

Estas nuevas tecnologías digitales que permiten la creación de contenido
generado por el usuario han dado lugar a una tendencia hacia los microformatos
(''microformats''), es decir, información breve, sencilla y específica. Junto con
los sistemas de publicación personal, como blogs o wikis, se ha vuelto bastante
fácil para cualquiera crear su propio contenido, incluido el microcontenido
(''microcontent''). El microcontenido, es decir, ''información publicada en formato
breve'', se relaciona más con un "enfoque formal de cómo presentar" el contenido
que con la calidad inherente del contenido en sí. Algunos ejemplos de
microcontenido son los podcasts, las entradas de blog, las páginas wiki o los
mensajes cortos en Facebook o X. La creación, publicación y compartición de
microcontenido en la web abre nuevas posibilidades para formas de aprendizaje
implícitas, informales e incidentales, como el microaprendizaje
(''microlearning''), término que se refiere a actividades breves de aprendizaje
con microcontenido.
\cite{article:microlearning_buchem}

\subsection{Antecedentes}
% Esta sección es conocida como "Antecedentes" o "Estado del Arte".

% Por ejemplo, si se habla de los algoritmos de ruteo, en esa sección se
% detallarán cuáles son los últimos algoritmos implementados para la comunidad
% científica. En esta sección, se explica solamente lo que ya existe y se hace
% un resumen: entonces se incluyen muchas referencias y ninguna frase o párrafo
% sobre la idea que se va a proponer.

El aprendizaje electrónico (''e-learning'') se ha convertido en una fuerza importante en el
mercado de la educación superior, con un valor de 325 mil millones de dólares
para el 2025 (Shah, 2023). El microaprendizaje consiste en enseñar contenido breve
pero lógicamente completo (Jomah et al., 2016). Según el informe de Clark et al.
(2023), la creciente tendencia del microaprendizaje surgió para cubrir la brecha
entre la creciente demanda de mano de obra cualificada por parte de las
industrias y la continua incapacidad de la educación superior para ofrecer
exalumnos "listos para el rendimiento". La pandemia de COVID-19 contribuyó a
esta tendencia desde una perspectiva diferente. Los estudiantes tenían
dificultades para comprender material complejo en casa sin interacciones
frecuentes de preguntas y respuestas (Wiley University Services, 2023). Dividir
el material en partes más pequeñas y fáciles de comprender con retroalimentación
rápida (pruebas) ayudó a los estudiantes a mantenerse atentos y motivados
durante largos períodos de distanciamiento social (Søzmen et al., 2023).

Leong et al. (2021) revisaron la literatura sobre microaprendizaje entre 2006 y
2019. Los autores identificaron 476 publicaciones relevantes sobre el tema.
Comenzó como una metodología de transferencia de conocimientos basada en el
trabajo, el aprendizaje. En la última década, el concepto de microaprendizaje se
ha vuelto ampliamente conocido y famoso. El informe del Sistema Universitario
Global de Canadá (2022) señaló las deficiencias de la educación presencial
tradicional y cómo la educación flexible y basada en habilidades puede
abordarlas. En busca de relevancia, las instituciones encontraron que el
microaprendizaje era relevante para su adopción en la educación superior (Leong
et al., 2021).

Shatte y Teague (2020) investigaron la literatura sobre microaprendizaje basado
en tecnología en la educación superior. Los autores muestran la contribución del
enfoque de microaprendizaje al aumento de los resultados educativos. Los
estudiantes reportaron una mayor motivación y un mayor compromiso con el
material y los procesos de aprendizaje al estudiar porciones pequeñas y concisas
de material (Shatte y Teague). Garshasbi et al. (2021) analizaron el efecto del
microaprendizaje en la enseñanza de las disciplinas STEM (ciencia, tecnología,
ingeniería y matemáticas). En un mundo con abundante información, la capacidad
de los estudiantes para concentrarse en clases virtuales, ya sea en línea o
presenciales, es cuestionable. El microaprendizaje podría ser una solución
adecuada para abordar este desafío. En la educación en línea, una pequeña
porción de material, con un enfoque lógico y concluyente, preparada para su
entrega o consumo por parte de los estudiantes ha demostrado ser una forma
eficaz de transferencia de conocimiento (Garshasbi et al.).
\cite{article:elearning_future_trends_shahid}

\subsection{Idea}
En este trabajo, vamos a desarrollar una solución web de capacitación empresarial que incorpora el microaprendizaje
personalizado para optimizar el tiempo que los empleados dedican al mismo.
La misma, busca una experiencia de capacitación flexible y personalizada, adaptándose al tiempo disponible de cada empleado.
Además, las rutas de aprendizaje se ajustan a los objetivos individuales, fomentando el desarrollo profesional de manera constante.
El objetivo del paper es analizar cómo esta plataforma, mediante actividades breves como videos, lecturas y cuestionarios,
logra un aprendizaje eficiente adaptado a las necesidades individuales de los empleados.
En este artículo, se explican los beneficios de la accesibilidad multiplataforma, que garantiza continuidad en el proceso formativo,
y los efectos positivos que este enfoque ha demostrado tener en la eficiencia del aprendizaje y el crecimiento profesional.

\subsection{Secciones}

En \textbf{Introducción}, se plantea la problemática, se contextualiza con antecedentes y se presenta la idea central del trabajo.
En \textbf{Objetivos} se define las metas de la investigación, centrándose en mejorar la eficiencia del aprendizaje corporativo.
En \textbf{Justificación}, se argumenta la relevancia del enfoque, respaldado por estudios previos.
En \textbf{Marco Teórico} se profundizan los conceptos clave y marca diferencias con otros enfoques educativos.
En \textbf{Trabajos Relacionados} se revisan herramientas similares y metodologías aplicadas en otros entornos empresariales.
En \textbf{Enfoque Propuesto} se describe la solución a ser desarrollada y la arquitectura de software a ser utilizada.
En \textbf{Diseño de la Investigación} se detalla la metodología utilizada para evaluar la efectividad del sistema.
En \textbf{Instrumentos} se presentan las herramientas utilizadas, como frameworks tecnológicos y plataformas de desarrollo.
En \textbf{Componentes} se listan los elementos clave del sistema a nivel software. Se vuelve a mencionar la arquitectura pero desde la implementación.
En \textbf{Procedimientos} se explica los pasos para la implementación de la solución.
En \textbf{Resultados} se analiza el impacto del sistema, describiendo métricas y pruebas realizadas.
En \textbf{Conclusión}, se evalúan los hallazgos obtenidos y se destacan las contribuciones del modelo de microaprendizaje personalizado.
En \textbf{Apéndices} se recopila información complementaria relevante.
En \textbf{Reconocimientos}, se agradecen contribuciones significativas al proyecto.
En \textbf{Referencias} se documentan fuentes y estudios clave que sustentan la investigación.
En \textbf{Biografía}, se presenta el perfil del autor.
