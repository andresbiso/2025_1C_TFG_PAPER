\IEEEPARstart{E}{ste} paper desarrolla la siguiente idea.

\subsection{Idea}
Una solución web de capacitación empresarial que incluye microaprendizaje personalizado con el objetivo de permitir
a los empleados aprovechar eficientemente su tiempo disponible.

Esta solución web ofrece una experiencia de capacitación flexible y personalizada,
adaptándose al tiempo disponible de cada empleado.
Mediante el microaprendizaje, proporciona actividades breves como videos,
lecturas o cuestionarios que maximizan el aprendizaje en períodos cortos,
como 15 minutos. Además, las rutas de aprendizaje se ajustan a los objetivos y necesidades individuales,
fomentando el desarrollo profesional de manera constante. La plataforma es accesible desde cualquier dispositivo,
promoviendo la comodidad y la continuidad en el proceso de capacitación y garantizando un impacto significativo en la
formación empresarial.

\subsection{Microaprendizaje}
El microaprendizaje ha evolucionado debido a la necesidad de centrarse menos en las nuevas tecnologías en sí y más en
las necesidades individuales de aprendizaje (Chisholm, 2005; Robes, 2009). El microaprendizaje se refiere a formas breves
de aprendizaje y consiste en actividades de aprendizaje breves, detalladas, interconectadas y poco acopladas,
con microcontenido (Lindner, 2006; Schmidt, 2007).
\cite{article:microlearning_buchem}

% ¿Qué escribir en esta sección?
% Motivación, se incluyen los antecedentes y una descripción muy detallada de la idea.
% Esta sección y marco teórico deben contener casi todas las referencias del paper.

% ¿Cómo debe escribirse esta sección?
% Abarca mínimo media página (idealmente debería finalizar casi al final de la página 1).
