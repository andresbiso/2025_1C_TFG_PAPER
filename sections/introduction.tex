% Sección obligatoria.
% ¿Qué escribir en esta sección?
% Motivación (o problemática), se incluyen los antecedentes y una descripción técnica y detallada de la idea.
% Por último, se debe incluir una descripción del resto de las secciones del paper.
% Esta sección y marco teórico deben contener casi todas las referencias del paper.

% La problemática o motivación del paper es el inicio de la introducción y sirve para saber si el problema que plantea el paper es conocido,
% investigado y cuál es su impacto en el mundo real.
% En esa parte de la introducción es necesario incluir referencias para identificar la problemática.

% Luego de plantear la problemática, es necesario encuadrar las soluciones ya existentes para el problema así como la nueva tecnología que podría ayudar a
% solucionar el tema.
% En la introducción no se explica cómo funcionan los algoritmos, solamente se citan para entender la complejidad.

% La idea debe comenzar con: “En este trabajo, vamos a desarrollar…”, “El objetivo del paper es…”, “En este artículo se explican…"
% Escribir ¿De qué se trata su trabajo?

% La descripción de las secciones debe ser un parrafo con una descripción breve de las secciones.

% ¿Cómo debe escribirse esta sección?
% Abarca mínimo media página (idealmente debería finalizar casi al final de la página 1).

% \IEEEPARstart{E}{ste} paper desarrolla la siguiente idea.

\subsection{Problemática}
\blindtext

% Esta sección es conocida como "Motivación" o "Problemática".

\subsection{Antecedentes}
\blindtext
% Esta sección es conocida como "Antecedentes" o "Estado del Arte".

% Por ejemplo, si se habla de los algoritmos de ruteo, en esa sección se detallarán cuáles son los últimos algoritmos implementados para la comunidad científica.
% En esta sección, se explica solamente lo que ya existe y se hace un resumen:
% entonces se incluyen muchas referencias y ninguna frase o párrafo sobre la idea que se va a proponer.

\subsection{Idea}
En este trabajo, vamos a desarrollar una solución web de capacitación empresarial que incorpora el microaprendizaje
personalizado para optimizar el tiempo que los empleados dedican al mismo.
La misma, busca una experiencia de capacitación flexible y personalizada, adaptándose al tiempo disponible de cada empleado.
Además, las rutas de aprendizaje se ajustan a los objetivos individuales, fomentando el desarrollo profesional de manera constante.
El objetivo del paper es analizar cómo esta plataforma, mediante actividades breves como videos, lecturas y cuestionarios,
logra un aprendizaje eficiente adaptado a las necesidades individuales de los empleados.
En este artículo, se explican los beneficios de la accesibilidad multiplataforma, que garantiza continuidad en el proceso formativo,
y los efectos positivos que este enfoque ha demostrado tener en la eficiencia del aprendizaje y el crecimiento profesional.

\subsection{Secciones}

En \textbf{Introducción}, se plantea la problemática, se contextualiza con antecedentes y se presenta la idea central del trabajo.
En \textbf{Objetivos} se define las metas de la investigación, centrándose en mejorar la eficiencia del aprendizaje corporativo.
En \textbf{Justificación}, se argumenta la relevancia del enfoque, respaldado por estudios previos.
En \textbf{Marco Teórico} se profundizan los conceptos clave y marca diferencias con otros enfoques educativos.
En \textbf{Trabajos Relacionados} se revisan herramientas similares y metodologías aplicadas en otros entornos empresariales.
En \textbf{Enfoque Propuesto} se describe la solución a ser desarrollada y la arquitectura de software a ser utilizada.
En \textbf{Diseño de la Investigación} se detalla la metodología utilizada para evaluar la efectividad del sistema.
En \textbf{Instrumentos} se presentan las herramientas utilizadas, como frameworks tecnológicos y plataformas de desarrollo.
En \textbf{Componentes} se listan los elementos clave del sistema a nivel software. Se vuelve a mencionar la arquitectura pero desde la implementación.
En \textbf{Procedimientos} se explica los pasos para la implementación de la solución.
En \textbf{Resultados} se analiza el impacto del sistema, describiendo métricas y pruebas realizadas.
En \textbf{Conclusión}, se evalúan los hallazgos obtenidos y se destacan las contribuciones del modelo de microaprendizaje personalizado.
En \textbf{Apéndices} se recopila información complementaria relevante.
En \textbf{Reconocimientos}, se agradecen contribuciones significativas al proyecto.
En \textbf{Referencias} se documentan fuentes y estudios clave que sustentan la investigación.
En \textbf{Biografía}, se presenta el perfil del autor.
