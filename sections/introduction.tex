\IEEEPARstart{E}{ste} paper desarrolla la siguiente idea.

\subsection{Idea}
Una solución web de capacitación empresarial que incluye microaprendizaje personalizado con el objetivo de permitir
a los empleados aprovechar eficientemente su tiempo disponible.

Esta solución web ofrece una experiencia de capacitación flexible y personalizada,
adaptándose al tiempo disponible de cada empleado.
Mediante el microaprendizaje, proporciona actividades breves como videos,
lecturas o cuestionarios que maximizan el aprendizaje en períodos cortos,
como 15 minutos. Además, las rutas de aprendizaje se ajustan a los objetivos y necesidades individuales,
fomentando el desarrollo profesional de manera constante. La plataforma es accesible desde cualquier dispositivo,
promoviendo la comodidad y la continuidad en el proceso de capacitación y garantizando un impacto significativo en la
formación empresarial.

\subsection{Microaprendizaje}
El microaprendizaje ha evolucionado debido a la necesidad de centrarse menos en las nuevas tecnologías en sí y más en
las necesidades individuales de aprendizaje (Chisholm, 2005; Robes, 2009). El microaprendizaje se refiere a formas breves
de aprendizaje y consiste en actividades de aprendizaje breves, detalladas, interconectadas y poco acopladas,
con microcontenido (Lindner, 2006; Schmidt, 2007).
\cite{article:microlearning_buchem}

% Sección obligatoria.
% ¿Qué escribir en esta sección?
% Motivación (o problemática), se incluyen los antecedentes y una descripción técnica y detallada de la idea.
% Por último, se debe incluir una descripción del resto de las secciones del paper.
% Esta sección y marco teórico deben contener casi todas las referencias del paper.

% La problemática o motivación del paper es el inicio de la introducción y sirve para saber si el problema que plantea el paper es conocido,
% investigado y cuál es su impacto en el mundo real.
% En esa parte de la introducción es necesario incluir referencias para identificar la problemática.

% Luego de plantear la problemática, es necesario encuadrar las soluciones ya existentes para el problema así como la nueva tecnología que podría ayudar a
% solucionar el tema.
% En la introducción no se explica cómo funcionan los algoritmos, solamente se citan para entender la complejidad.

% La idea debe comenzar con: “En este trabajo, vamos a desarrollar…”, “El objetivo del paper es…”, “En este artículo se explican…
% Escribir ¿De qué se trata su trabajo?

% La descripción de las secciones debe ser un parrafo con una descripción breve de las secciones.

% ¿Cómo debe escribirse esta sección?
% Abarca mínimo media página (idealmente debería finalizar casi al final de la página 1).

\subsection{Problemática}
\blindtext

% Esta sección es conocida como "Motivación" o "Problemática".

\subsection{Antecedentes}
\blindtext
% Esta sección es conocida como "Antecedentes" o "Estado del Arte".

% Por ejemplo, si se habla de los algoritmos de ruteo, en esa sección se detallarán cuáles son los últimos algoritmos implementados para la comunidad científica.
% En esta sección, se explica solamente lo que ya existe y se hace un resumen:
% entonces se incluyen muchas referencias y ninguna frase o párrafo sobre la idea que se va a proponer.



\subsection{Idea}

\subsection{Secciones}
