% Sección obligatoria.
% ¿Cómo debe escribirse esta sección?
% Está formado por dos párrafos, uno para la motivación/problemática y otro para
% la idea, donde se contextualiza la investigación del paper. La idea debe estar
% escrita desde un punto de vista técnico. Debe contener la conclusión de la
% investigación (forma resumida).
\begin{abstract}
    En la actualidad, la falta de tiempo disponible representa un desafío para
    la educación. En el entorno laboral, las personas deben equilibrar múltiples
    responsabilidades, lo que limita su capacidad para participar en procesos de
    formación extensos. A medida que la tecnología, la economía y la sociedad
    evolucionan rápidamente, surge la necesidad de enfoques educativos
    alternativos, que permitan adquirir conocimientos en períodos cortos sin
    comprometer su calidad. En este contexto, el microaprendizaje ofrece una
    solución flexible y accesible, facilitando el acceso a información concisa y
    adaptada a las necesidades individuales de cada empleado.

    Este trabajo presenta el desarrollo de una solución web de capacitación
    empresarial basada en microaprendizaje, diseñada para optimizar el tiempo de
    los empleados sin afectar la continuidad de su formación. La solución adapta
    los contenidos según el tiempo disponible y los objetivos de cada usuario,
    promoviendo un desarrollo profesional constante. A través de actividades
    breves, como videos, lecturas y cuestionarios interactivos, se garantiza un
    aprendizaje eficiente y accesible desde cualquier dispositivo.
    
    Los resultados obtenidos demuestran que la implementación de esta solución
    ha mejorado significativamente la retención del conocimiento y la aplicación
    de habilidades en el entorno laboral. La personalización del contenido y la
    accesibilidad multiplataforma han favorecido la continuidad en la formación,
    permitiendo que los empleados progresen en sus rutas de aprendizaje sin
    interrupciones. Además, se ha observado un impacto positivo en la eficiencia
    del aprendizaje y el crecimiento profesional, evidenciando que el
    microaprendizaje es una estrategia efectiva para la capacitación
    empresarial.
\end{abstract}