\begin{abstract} 
    En este trabajo, se desarrolla una solución web de capacitación empresarial basada en microaprendizaje
    personalizado para optimizar el tiempo que los empleados dedican al aprendizaje.
    La plataforma ofrece una experiencia flexible, adaptándose al tiempo disponible de cada usuario y ajustando las rutas
    de aprendizaje a sus objetivos individuales, promoviendo el desarrollo profesional constante.
    A través de actividades breves como videos, lecturas y cuestionarios, se logra un aprendizaje eficiente,
    accesible desde cualquier dispositivo. Los resultados han demostrado que este enfoque mejora la continuidad en la formación
    y tiene un impacto positivo en la eficiencia del aprendizaje y el crecimiento profesional.
\end{abstract}

% Sección obligatoria.
% ¿Cómo debe escribirse esta sección?
% Está formado por dos párrafos, uno para la motivación/problemática y otro para la idea, donde se contextualiza la investigación del paper.
% La idea debe estar escrita desde un punto de vista técnico. Debe contener la conclusión de la investigación (forma resumida).