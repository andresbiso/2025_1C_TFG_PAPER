% ¿Qué escribir en esta sección?
% Esta sección e introducción deben contener casi todas las referencias del paper.
% Acá tenemos que identificar los conceptos básicos de la investigación que serán empleados en las secciones sucesivas.
% En esta sección se orienta la investigación desde un punto de vista innovador marcando las posibles diferencias con otras investigaciones.

\subsection{Microaprendizaje}
El microaprendizaje se refiere a formas breves de aprendizaje y consiste en
actividades de aprendizaje breves, detalladas, interconectadas y poco acopladas,
con microcontenido (Lindner, 2006; Schmidt, 2007).
\cite{article:microlearning_buchem}

\subsection{Aprendizaje personalizado (AP)}
El aprendizaje personalizado (AP) es un entorno de aprendizaje digitalizado
diseñado para satisfacer los resultados de aprendizaje de un individuo y
adaptado a su conocimiento y experiencia.
\cite{article:elearning_future_trends_shahid}

\subsection{Aprendizaje adaptativo (AA)}
Término que describe una situación en la que los conocimientos impartidos se
adaptan a las necesidades de cada estudiante.
\cite{article:elearning_future_trends_shahid}