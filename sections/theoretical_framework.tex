% ¿Qué escribir en esta sección?
% Esta sección e introducción deben contener casi todas las referencias del paper.
% Acá tenemos que identificar los conceptos básicos de la investigación que serán empleados en las secciones sucesivas.
% En esta sección se orienta la investigación desde un punto de vista innovador marcando las posibles diferencias con otras investigaciones.

\subsection{Microaprendizaje}
\subsubsection{Definición}
El microaprendizaje se refiere a formas breves de aprendizaje y consiste en
actividades de aprendizaje breves, detalladas, interconectadas y poco acopladas,
con microcontenido (Lindner, 2006; Schmidt, 2007).
\cite{article:microlearning_buchem}

\subsubsection{Características del Microaprendizaje}
\paragraph{\textbf{El microaprendizaje es breve}}

Aunque hay debate sobre la duración óptima de una actividad de microaprendizaje,
la mayoría de los estudios sugieren que una duración de unos pocos minutos es lo
habitual.
\cite{article:when_i_say_microlearning}

\paragraph{\textbf{El microaprendizaje se centra en un único objetivo de aprendizaje}}
La pieza de información o el objetivo de aprendizaje que se entrega a los
estudiantes a veces se denomina fragmento de información, unidad o microunidad.
Para evitar la sobrecarga cognitiva, no se proporciona información adicional
sobre el tema. Cualquier dato adicional solo generaría distracción y
dificultaría el aprendizaje.
\cite{article:when_i_say_microlearning}

\paragraph{\textbf{El microaprendizaje puede lograrse en cualquier momento que
el estudiante lo necesite}}

Es asíncrono, lo que significa que cada estudiante puede acceder a los recursos
en el momento y lugar que prefiera. Esto facilita un aprendizaje justo a tiempo.
\cite{article:when_i_say_microlearning}

\paragraph{\textbf{Los recursos de microaprendizaje se entregan comúnmente
mediante tecnología digital (aunque el microaprendizaje no está definido por la
tecnología)}}

Los medios para el microaprendizaje incluyen presentaciones, documentos PDF,
podcasts, infografías, cuestionarios, videos, módulos de aprendizaje en línea,
blogs y aplicaciones de redes sociales. Muchos de estos medios también son
objetos de aprendizaje reutilizables y auto-contenidos. 

El teléfono inteligente conectado a internet es un impulsor clave del
microaprendizaje, ya que permite el acceso al aprendizaje en movimiento
(aprendizaje móvil) y está casi siempre disponible para la mayoría de las
personas. Independientemente del medio utilizado, debe emplearse para potenciar
las características fundamentales del microaprendizaje: breve, centrado en un
solo objetivo de aprendizaje y accesible en cualquier momento y lugar.

El microaprendizaje en aplicaciones Web 2.0 también ofrece el beneficio de un
espacio de aprendizaje colaborativo. Web 2.0 se refiere a los sitios web que
admiten redes sociales y fomentan la generación de contenido, la interacción y
la participación de los usuarios. De esta manera, el estudiante se convierte en
un aprendiz autodeterminado y activamente comprometido, contribuyendo a la
creación de contenido educativo en un enfoque conocido como heutagogía,
facilitando el aprendizaje entre pares.
\cite{article:when_i_say_microlearning}

\subsection{Aprendizaje personalizado (AP)}
El aprendizaje personalizado (AP) es un entorno de aprendizaje digitalizado
diseñado para satisfacer los resultados de aprendizaje de un individuo y
adaptado a su conocimiento y experiencia.
\cite{article:elearning_future_trends_shahid}

\subsection{Aprendizaje adaptativo (AA)}
Término que describe una situación en la que los conocimientos impartidos se
adaptan a las necesidades de cada estudiante.
\cite{article:elearning_future_trends_shahid}