Esta sección describe la metodología utilizada para evaluar la
factibilidad del diseño de un Learning Management System (LMS) basado
en microaprendizaje con integración a plataformas de mensajería. Se analiza la
viabilidad técnica y estructural del sistema, considerando aspectos clave como
la arquitectura, la escalabilidad y la integración con herramientas externas.  

\subsection{Metodología}  
Para determinar la factibilidad del diseño, se lleva a cabo un análisis
exploratorio basado en pruebas de concepto y revisión de modelos previos. La
metodología empleada abarca los siguientes aspectos:  

\begin{itemize}  
    \item \textbf{Evaluación tecnológica}: Se examinan las tecnologías
    existentes que permiten la implementación de un LMS con soporte para
    microaprendizaje, identificando su compatibilidad con arquitecturas
    cliente-servidor y la integración con plataformas externas.  
    \item \textbf{Análisis arquitectónico}: Se diseña un modelo conceptual de la
    arquitectura del sistema, fundamentado en enfoques propuestos en trabajos
    previos.  
    \item \textbf{Pruebas de integración}: Se exploran mecanismos de
    comunicación entre el LMS y plataformas de mensajería como Telegram,
    verificando su viabilidad dentro del diseño.  
    \item \textbf{Escalabilidad y adaptabilidad}: Se evalua la capacidad del
    sistema para ajustarse a distintos escenarios empresariales y soportar una
    expansión futura.  
\end{itemize}  

\subsection{Procedimiento}  
El análisis de factibilidad se lleva a cabo mediante las siguientes etapas:  

\begin{enumerate}  
    \item \textbf{Revisión de literatura}: Se investigan sistemas LMS con
    características similares, identificando fortalezas y limitaciones.  
    \item \textbf{Diseño conceptual}: Se definen los módulos clave y su
    interacción dentro de la arquitectura propuesta.  
    \item \textbf{Evaluación de integración}: Se prueban mecanismos para la
    comunicación entre el LMS y Telegram, verificando la posibilidad de
    automatizar notificaciones.  
    \item \textbf{Análisis de viabilidad}: Se identifican potenciales desafíos
    técnicos y soluciones para asegurar la implementación del sistema en un
    entorno empresarial.  
\end{enumerate}  

Este enfoque metodológico pertmite obtener un marco claro sobre la viabilidad
del sistema propuesto, proporcionando las bases necesarias para futuras fases de
desarrollo e implementación.  
