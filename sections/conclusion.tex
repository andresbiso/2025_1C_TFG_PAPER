% Sección obligatoria.
% ¿Qué escribir en esta sección?
% Reafirmar la idea en este apartado. Una versión simple.
% Solamente informamos si los resultados fueron positivos o negativos. Si resultó exitosa o no.
% A su vez en la conclusión también es necesario explicar los obstáculos del paper e indicar los posibles trabajos futuros.

% ¿Cómo debe escribirse esta sección?
% No tiene extensión definida, puede ser tan extensa como se requiera

Este trabajo demuestra que la solución web de capacitación empresarial basada en
microaprendizaje personalizado es una estrategia efectiva para mejorar el
aprendizaje y el desarrollo profesional. Adaptando el contenido a las
necesidades individuales y facilitando el acceso desde cualquier dispositivo, se
ha logrado optimizar el tiempo de los empleados, garantizando un proceso de
formación continuo y con impacto significativo en el ámbito laboral.

En un entorno laboral dinámico y en constante evolución, es fundamental que las
competencias clave se adquieran, actualicen y mantengan de manera permanente.
Los métodos tradicionales de formación suelen ser rígidos y poco eficientes para
responder a las necesidades cambiantes del mundo profesional. En contraste, el
microaprendizaje, integrado en plataformas digitales como la solución web
presentada, permite una capacitación ágil y accesible. A través de contenidos
breves y personalizados, combinados con accesibilidad multiplataforma, esta
metodología facilita el aprendizaje progresivo, adaptado a los ritmos de trabajo
y promoviendo el desarrollo profesional continuo.