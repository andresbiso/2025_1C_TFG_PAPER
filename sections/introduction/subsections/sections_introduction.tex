En \textbf{Introducción}, se plantea la problemática, se define el objetivo y las metas de la investigación, se contextualiza con antecedentes y se presenta la idea central del trabajo.
En \textbf{Justificación}, se argumenta la relevancia del enfoque, respaldado por estudios previos.
En \textbf{Marco Teórico} se profundizan los conceptos clave y marca diferencias con otros enfoques educativos.
En \textbf{Trabajos Relacionados} se revisan herramientas similares y metodologías aplicadas en otros entornos empresariales.
En \textbf{Enfoque Propuesto} se describe la solución a ser desarrollada y la arquitectura de software a ser utilizada.
En \textbf{Diseño de la Investigación} se detalla la metodología utilizada para evaluar la efectividad del sistema.
En \textbf{Instrumentos} se presentan las herramientas utilizadas, como frameworks tecnológicos y plataformas de desarrollo.
En \textbf{Componentes} se listan los elementos clave del sistema a nivel software. Se vuelve a mencionar la arquitectura pero desde la implementación.
En \textbf{Procedimientos} se explica los pasos para la implementación de la solución.
En \textbf{Resultados} se analiza el impacto del sistema y se describen las pruebas realizadas.
En \textbf{Conclusión}, se evalúan los hallazgos obtenidos y se destacan las contribuciones del modelo de microaprendizaje.
En \textbf{Apéndices} se recopila información complementaria relevante.
En \textbf{Reconocimientos}, se agradecen contribuciones significativas al proyecto.
En \textbf{Referencias} se documentan fuentes y estudios clave que sustentan la investigación.
En \textbf{Biografía}, se presenta el perfil del autor.