% Esta sección es conocida como ''Antecedentes'' o ''Estado del Arte''.

% Por ejemplo, si se habla de los algoritmos de ruteo, en esa sección se
% detallarán cuáles son los últimos algoritmos implementados para la comunidad
% científica. En esta sección, se explica solamente lo que ya existe y se hace
% un resumen: entonces se incluyen muchas referencias y ninguna frase o párrafo
% sobre la idea que se va a proponer.
El aprendizaje electrónico se ha convertido en una fuerza importante en el
mercado de la educación superior, con un valor de 325 000 millones de dólares
para 2025 (Shah, 2023). El microaprendizaje consiste en enseñar contenido breve
pero lógicamente completo (Jomah et al., 2016). Según el informe de Clark et al.
(2023), la creciente tendencia del microaprendizaje surgió para cubrir la
necesidad de una fuerza laboral cualificada en las industrias en crecimiento y
la continua incapacidad de la educación superior para ofrecer egresados
''listos para el desempeño''.

El aprendizaje electrónico (''e-learning'') se ha convertido en una fuerza importante en el
mercado de la educación superior, con un valor de 325 mil millones de dólares
para el 2025 (Shah, 2023). El microaprendizaje consiste en enseñar contenido breve
pero lógicamente completo (Jomah et al., 2016). Según el informe de Clark et al.
(2023), la creciente tendencia del microaprendizaje surgió para cubrir la brecha
entre la creciente demanda de mano de obra cualificada por parte de las
industrias y la continua incapacidad de la educación superior para ofrecer
egresados ''listos para el desempeño''. La pandemia de COVID-19 contribuyó a
esta tendencia desde una perspectiva diferente. Los estudiantes tenían
dificultades para comprender material complejo en casa sin interacciones
frecuentes de preguntas y respuestas (Wiley University Services, 2023). Dividir
el material en partes más pequeñas y fáciles de comprender con retroalimentación
rápida (pruebas) ayudó a los estudiantes a mantenerse atentos y motivados
durante largos períodos de distanciamiento social (Søzmen et al., 2023).

Leong et al. (2021) revisaron la literatura sobre microaprendizaje entre 2006 y
2019. Los autores identificaron 476 publicaciones relevantes sobre el tema.
Comenzó como una metodología de transferencia de conocimientos basada en el
trabajo, el aprendizaje. En la última década, el concepto de microaprendizaje se
ha vuelto ampliamente conocido y famoso. El informe del Sistema Universitario
Global de Canadá (2022) señaló las deficiencias de la educación presencial
tradicional y cómo la educación flexible y basada en habilidades puede
abordarlas. En busca de relevancia, las instituciones encontraron que el
microaprendizaje era relevante para su adopción en la educación superior (Leong
et al., 2021).

Shatte y Teague (2020) investigaron la literatura sobre microaprendizaje basado
en tecnología en la educación superior. Los autores muestran la contribución del
enfoque de microaprendizaje al aumento de los resultados educativos. Los
estudiantes reportaron una mayor motivación y un mayor compromiso con el
material y los procesos de aprendizaje al estudiar porciones pequeñas y concisas
de material (Shatte y Teague). Garshasbi et al. (2021) analizaron el efecto del
microaprendizaje en la enseñanza de las disciplinas STEM (ciencia, tecnología,
ingeniería y matemáticas). En un mundo con abundante información, la capacidad
de los estudiantes para concentrarse en clases virtuales, ya sea en línea o
presenciales, es cuestionable. El microaprendizaje podría ser una solución
adecuada para abordar este desafío. En la educación en línea, una pequeña
porción de material, con un enfoque lógico y concluyente, preparada para su
entrega o consumo por parte de los estudiantes ha demostrado ser una forma
eficaz de transferencia de conocimiento (Garshasbi et al.).
\cite{article:elearning_future_trends_shahid}