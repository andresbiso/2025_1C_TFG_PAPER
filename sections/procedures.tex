Esta sección describe los pasos seguidos para la implementación de la solución
LMS basada en microaprendizaje, asegurando un desarrollo estructurado y
eficiente.  

\subsection{Configuración de la Arquitectura}  
Para la implementación del sistema, se define una arquitectura cliente-servidor
compuesta por varios módulos clave. Se diseñan los componentes principales
para garantizar una interacción fluida entre los usuarios y el sistema,
asegurando accesibilidad y escalabilidad.  

\subsection{Implementación del Cliente}  
La interfaz de usuario es desarrollada utilizando React, optimizada
con Vite para mejorar el rendimiento. Los pasos de implementación
incluyen:  

\begin{itemize}  
    \item Configuración del entorno de desarrollo con React.  
    \item Desarrollo de componentes dinámicos para la presentación de cursos.  
    \item Integración con API REST para la comunicación con el backend.  
    \item Gestión del estado y almacenamiento local de preferencias del usuario.  
\end{itemize}  

\subsection{Desarrollo del Servidor}  
El backend, construido en Node.js con Express, se encarga de gestionar
la lógica de negocio y la seguridad del sistema. Su implementación se realiza
siguiendo los siguientes pasos:  

\begin{itemize}  
    \item Creación de endpoints para la gestión de usuarios, cursos e
    integraciones.  
    \item Implementación de autenticación y autorización mediante JWT.  
    \item Integración con MongoDB para el almacenamiento de datos estructurados.  
    \item Comunicación con servicios externos provistos a través de contenedores
    Docker para garantizar escalabilidad.  
\end{itemize}  

\subsection{Integración con Telegram}  
Para mejorar la interacción con los usuarios, se desarrolla un bot de Telegram
que permite la notificación automática de nuevos cursos. La integración sigue
estos pasos:  

\begin{itemize}  
    \item Creación del bot utilizando la API de Telegram.  
    \item Desarrollo de un servidor intermedio en Flask para gestionar las
    solicitudes.  
    \item Configuración de comunicación entre el bot y el sistema LMS.  
\end{itemize}  

\subsection{Despliegue y Mantenimiento}  
Lo servicios, el bot server y el bot se configuran para ejecutarse en
contenedores Docker, asegurando portabilidad y facilidad de mantenimiento.
El servidor, por su parte, se despliega a un ambiente configurado para su
correcta ejecución.
Para el despliegue se llevan a cabo las siguientes acciones:  

\begin{itemize}  
    \item Creación de archivos Docker para cada módulo del sistema que haga uso del mismo.  
    \item Automatización de despliegues utilizando Docker Compose.  
\end{itemize}  

Este enfoque metodológico permite una implementación estructurada del sistema,
facilitando la optimización de sus funcionalidades y asegurando la integración
efectiva de microaprendizaje en un entorno empresarial.  
