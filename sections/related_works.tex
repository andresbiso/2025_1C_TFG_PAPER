% Los trabajos relacionados podrían ser herramientas que ya existen
% o que se mencionan en los paper que hacen parte de lo mencionado en el paper.

El microlearning ha sido ampliamente estudiado como una estrategia flexible y
efectiva para la educación y el desarrollo profesional. Diversas investigaciones
han explorado su integración con plataformas digitales y tecnologías emergentes,
proporcionando un marco teórico y práctico que respalda su implementación en
entornos corporativos.

A continuación, haré mención de estudios previos que han permitido comprender la
manera de implementar estrategias de aprendizaje informal y microlearning dentro
de un LMS. Manteniendo el objeto de identificar los avances, desafíos y
oportunidades en la formación empresarial basada en esta metodología.

Estos estudios constituyen una base fundamental para la implementación del
microlearning en entornos empresariales, ofreciendo un panorama amplio de sus
aplicaciones y beneficios.

La investigación ''Microlearning: a strategy for ongoing professional
development'' llevó a cabo el siguiente trabajo: ''Este artículo presenta el
microaprendizaje como un enfoque moderno y flexible para el aprendizaje a lo
largo de la vida que se integra perfectamente en la rutina diaria, influenciado
por tecnologías como Web 2.0 y software social. Explora los conceptos de
microcontenido y microaprendizaje en relación con el aprendizaje informal y
basado en el trabajo, destacando su relevancia para los estudiantes actuales. El
estudio describe principios clave de diseño, enfatizando la creación de
microcontenidos y actividades de microaprendizaje, e identifica diez
características que diferencian al microaprendizaje de los formatos
tradicionales de ''macroaprendizaje''. Al conectar la educación formal e informal,
el microaprendizaje, respaldado por tecnologías Web 2.0, ofrece una solución
dinámica para entornos de aprendizaje acelerados y multitarea, apoyando el
desarrollo profesional continuo.''
\cite{article:microlearning_buchem}

La investigación ''Microlearning: Knowledge Management Applications and
Competency-Based Training in the Workplace'' llevó a cabo el siguiente trabajo:
''El enfoque de este artículo es una discusión triple sobre el microaprendizaje:
Cómo las mejores prácticas del microaprendizaje facilitan la adquisición de
conocimientos en el lugar de trabajo al involucrar y motivar a los empleados
mediante un aprendizaje breve, personalizado y justo a tiempo.
De qué manera el microaprendizaje se integra con las aplicaciones de gestión del
conocimiento a través del mentoría situacional.
Cómo el microaprendizaje basado en competencias, mediante el aprendizaje por
suscripción, es tanto un enfoque innovador para el aprendizaje en línea como un
recurso valioso para las organizaciones de aprendizaje centradas en mejorar el
rendimiento de sus empleados.''
\cite{article:microlearning_emerson}

La investigación ''Informal Learning in the Workplace'' llevó a cabo el siguiente
trabajo:
''Este artículo se centra principalmente en los marcos teóricos para comprender e
investigar el aprendizaje informal en el lugar de trabajo, desarrollados a
través de una serie de proyectos de gran y pequeña escala. Se incluyen las
principales conclusiones, pero se remite a los lectores a otras publicaciones
para obtener descripciones más detalladas de proyectos individuales. Se discuten
dos tipos de marco. El primer grupo busca deconstruir los ''conceptos clave'' del
aprendizaje informal, el aprendizaje de la experiencia, el conocimiento tácito,
la transferencia del aprendizaje y la práctica intuitiva para revelar la gama de
diferentes fenómenos que abarcan estos términos populares. El segundo grupo
comprende marcos para abordar las tres preguntas centrales que impregnaron el
programa de investigación: ¿qué se está aprendiendo, cómo se está aprendiendo y
cuáles son los factores que influyen en el nivel y las direcciones del esfuerzo
de aprendizaje?''
\cite{article:informal_learning_eraut}