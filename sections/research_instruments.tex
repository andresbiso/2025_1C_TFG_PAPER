% La sección de Instrumentos contiene: github, editor de texto, herramientas
% utilizadas, frameworks,…
\subsection{Implementación y Herramientas Tecnológicas}

El desarrollo de  la \textit{"propuesta de solución de LMS orientado al
microaprendizaje para su uso en organizaciones"} se llevó a cabo bajo un entorno
estructurado que garantizó una implementación modular y escalable. A
continuación, se detallan las principales herramientas empleadas.

\subsection{Entorno de Desarrollo}
El sistema fue desarrollado en \textbf{macOS Sequoia 15 (x64, Intel)},
utilizando \textit{Homebrew} como gestor de paquetes. Este entorno proporcionó
estabilidad y compatibilidad con las tecnologías seleccionadas.

\subsection{Lenguajes y Frameworks}
Para la implementación de los diferentes módulos del sistema, se utilizaron las
siguientes tecnologías:
\begin{itemize}
	\item \textbf{Node.js} con \textit{Express.js}: Desarrollo del servidor
	backend.
	\item \textbf{Python} con \textit{Flask}: Implementación de la lógica del
	bot y su servidor.
	\item \textbf{React.js}: Desarrollo de la interfaz de usuario del cliente.
\end{itemize}

\subsection{Contenedores y Orquestación}
Con el objetivo de garantizar la modularidad y despliegue eficiente de los
servicios, se utilizó \textbf{Docker} en conjunto con \textbf{Docker Compose}.
Esto permitió la ejecución de componentes en entornos aislados y replicables.

\subsection{Servicios Adicionales}
Se integraron diversas herramientas para la gestión de datos y almacenamiento:
\begin{itemize}
	\item \textbf{MongoDB \& Mongo Express}: Base de datos NoSQL y herramienta
	de administración.
	\item \textbf{MinIO (API \& WebUI)}: Almacenamiento distribuido de objetos.
	\item \textbf{Mailpit (SMTP \& WebUI)}: Servidor de pruebas de correo
	electrónico.
	\item \textbf{DiceBear}: Generador de avatares dinámicos.
\end{itemize}

\subsection{Implementación y Despliegue}
La ejecución del sistema sigue un flujo estructurado que involucra la
instalación de dependencias, la configuración de contenedores y la puesta en
marcha de servicios. 

Para la gestión de la base de datos, se empleó \textbf{MongoDB Compass} y
\textbf{MongoDB Express}.

\subsection{Repositorio del Proyecto}
El código fuente y la documentación del proyecto están alojados en Github. Se
pueden encontrar en el siguiente repositorio:

\begin{center}
	\textbf{\href{https://github.com/andresbiso/2025\_1C\_TFG}{https://github.com/andresbiso/2025\_1C\_TFG}}
\end{center}
