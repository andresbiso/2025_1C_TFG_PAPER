% Este apéndice proporciona información complementaria para reforzar la solución
% propuesta sin sobrecargar el cuerpo principal del paper.
% - Especificaciones de software y hardware requeridas.
% - Diagramas de arquitectura del sistema LMS.
% - Algoritmos utilizados para personalizar el microaprendizaje.
% - Capturas de pantalla o ejemplos de la interfaz del LMS.
% - Resultados detallados de pruebas piloto o estudios de usabilidad.
% - Comparación con otros LMS tradicionales y sus limitaciones.
% - Formularios utilizados para evaluar la efectividad del sistema.
% - Respuestas clave de usuarios o docentes sobre la experiencia de aprendizaje.
% - Segmentos relevantes del código para la implementación del LMS.
% - Scripts utilizados para la automatización de contenido microlearning.
% - Fórmulas aplicadas en la recomendación de contenidos.
% - Estadísticas detalladas sobre mejora en el aprendizaje.
% - Glosario de términos técnicos utilizados.
% - Documentación técnica adicional.

% La opción [H] en \begin{figure}[H] proviene del paquete float y significa "Colocar la figura exactamente aquí".

\appendices
\section{Visualización de la interfaz y funcionalidades del sistema}
