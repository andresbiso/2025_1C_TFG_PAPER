\documentclass[journal]{IEEEtran}
\IEEEoverridecommandlockouts

\usepackage{algorithmic}
\usepackage{amsmath,amssymb,amsfonts}
\usepackage[english,spanish]{babel}
\usepackage{blindtext}
\usepackage{cite}
\usepackage{comment}
\usepackage{graphicx}
\usepackage[colorlinks=true, linkcolor=black, urlcolor=black, citecolor=black]{hyperref}
\usepackage{textcomp}
\usepackage{xcolor}

\graphicspath{ {images/} }
\def\BibTeX{{\rm B\kern-.05em{\sc i\kern-.025em b}\kern-.08em
    T\kern-.1667em\lower.7ex\hbox{E}\kern-.125emX}}

% thebibliography fix
\makeatletter
\def\endthebibliography{%
	\def\@noitemerr{\@latex@warning{Empty `thebibliography' environment}}%
	\endlist
}
\makeatother

\begin{document}
\selectlanguage{spanish}

% Comentarios
% - Las frases de sentido común están prohibidas
% - Una frase no referenciada significa que es un resultado de la investigación del autor.
% - Un argumento ya existente necesita solamente un resumen si es útil para la investigación. Ej: sección "tema específico".
% Todo referenciado con artículos o libros.
% - Cada frase, si no es el resultado de una investigación, necesita una referencia.
% - Si al finalizar el paper una sección no se utilizó, se puede quitar esa sección.
% - El paper se escribe en tiempo presente.
% - Si no es un resultado de la investigación, entonces es un dato y necesita una referencia.
% - Las secciones se pueden leer independientemente, sin saber las secciones anteriores o las siguientes. No referenciar secciones.
%- Evitar introducir un argumento o un concepto a nivel histórico. Poner directamente las fórmulas o el hecho necesario a la investigación.

\title{
    Microaprendizaje en organizaciones: Una propuesta de aprendizaje moderno y flexible\\
    %{\footnotesize \textsuperscript{*}...}
    %\thanks{...}
}

\author{
    \IEEEauthorblockN{Andrés Isaac Biso}
    \IEEEauthorblockA{
        \textit{Facultad de Ingeniería} \\
        \textit{Universidad de Palermo}\\
        Ciudad Autónoma de Buenos Aires, Argentina \\
    }
}
\maketitle

% Sección obligatoria.
% ¿Cómo debe escribirse esta sección?
% Está formado por dos párrafos, uno para la motivación/problemática y otro para
% la idea, donde se contextualiza la investigación del paper. La idea debe estar
% escrita desde un punto de vista técnico. Debe contener la conclusión de la
% investigación (forma resumida).
\begin{abstract}
    En la actualidad, la falta de tiempo disponible representa un desafío para
    la educación. En el entorno laboral, las personas deben equilibrar múltiples
    responsabilidades, lo que limita su capacidad para participar en procesos de
    formación extensos. A medida que la tecnología, la economía y la sociedad
    evolucionan rápidamente, surge la necesidad de enfoques educativos
    alternativos, que permitan adquirir conocimientos en períodos cortos sin
    comprometer su calidad. En este contexto, el microaprendizaje ofrece una
    solución flexible y accesible, facilitando el acceso a información concisa y
    adaptada a las necesidades individuales de cada empleado.

    Este trabajo presenta el desarrollo de una solución web de capacitación
    empresarial basada en microaprendizaje, diseñada para optimizar el tiempo de
    los empleados sin afectar la continuidad de su formación. La solución adapta
    los contenidos según el tiempo disponible y los objetivos de cada usuario,
    promoviendo un desarrollo profesional constante. A través de actividades
    breves, como videos, lecturas y cuestionarios interactivos, se garantiza un
    aprendizaje eficiente y accesible desde cualquier dispositivo.
    
    Los resultados obtenidos demuestran que la implementación de esta solución
    ha mejorado significativamente la retención del conocimiento y la aplicación
    de habilidades en el entorno laboral. La personalización del contenido y la
    accesibilidad multiplataforma han favorecido la continuidad en la formación,
    permitiendo que los empleados progresen en sus rutas de aprendizaje sin
    interrupciones. Además, se ha observado un impacto positivo en la eficiencia
    del aprendizaje y el crecimiento profesional, evidenciando que el
    microaprendizaje es una estrategia efectiva para la capacitación
    empresarial.
\end{abstract}

\begin{IEEEkeywords}
	informal learning, microlearning, microcontent, online communities, Web 2.0
\end{IEEEkeywords}

\section{Introduccíon}
% Sección obligatoria.
% ¿Qué escribir en esta sección?
% Motivación (o problemática), se incluyen los antecedentes y una descripción técnica y detallada de la idea.
% Por último, se debe incluir una descripción del resto de las secciones del paper.
% Esta sección y marco teórico deben contener casi todas las referencias del paper.

% La problemática o motivación del paper es el inicio de la introducción y sirve para saber si el problema
% que plantea el paper es conocido, investigado y cuál es su impacto en el mundo real.
% En esa parte de la introducción es necesario incluir referencias para identificar la problemática.

% Luego de plantear la problemática, es necesario encuadrar las soluciones ya
% existentes para el problema así como la nueva tecnología que podría ayudar a
% solucionar el tema.
% En la introducción no se explica cómo funcionan los algoritmos, solamente se citan para entender la complejidad.

% La idea debe comenzar con: ''En este trabajo, vamos a desarrollar…'', ''El objetivo del paper es…'', ''En este artículo se explican…''
% Escribir ¿De qué se trata su trabajo?

% La descripción de las secciones debe ser un parrafo con una descripción breve de las secciones.

% ¿Cómo debe escribirse esta sección?
% Abarca mínimo media página (idealmente debería finalizar casi al final de la página 1).

% \IEEEPARstart{E}{ste} paper desarrolla la siguiente idea.

\subsection{Problemática}
% Esta sección es conocida como "Motivación" o "Problemática".

En el mundo actual, las personas se enfrentan a una escasez de tiempo más que
nunca. La aparición de diversas preocupaciones ha limitado la posibilidad de un
enfoque a largo plazo. En esta situación, la manera más óptima de mejorar el sistema
educativo es utilizar un método que proporcione la mayor cantidad de información
en el menor tiempo posible.
\cite{article:microlearning_today_students_nikkhoo}

\subsection{Antecedentes}
% Esta sección es conocida como ''Antecedentes'' o ''Estado del Arte''.

% Por ejemplo, si se habla de los algoritmos de ruteo, en esa sección se
% detallarán cuáles son los últimos algoritmos implementados para la comunidad
% científica. En esta sección, se explica solamente lo que ya existe y se hace
% un resumen: entonces se incluyen muchas referencias y ninguna frase o párrafo
% sobre la idea que se va a proponer.
El aprendizaje electrónico (''e-learning'') se ha convertido en una fuerza importante en el
mercado de la educación superior, con un valor de 325 mil millones de dólares
para el 2025 (Shah, 2023). El microaprendizaje consiste en enseñar contenido breve
pero lógicamente completo (Jomah et al., 2016). Según el informe de Clark et al.
(2023), la creciente tendencia del microaprendizaje surgió para cubrir la brecha
entre la creciente demanda de mano de obra cualificada por parte de las
industrias y la continua incapacidad de la educación superior para ofrecer
egresados ''listos para el desempeño''.

La pandemia de COVID-19 contribuyó a esta tendencia desde una perspectiva
diferente. Los estudiantes tenían dificultades para comprender material complejo
en casa sin interacciones frecuentes de preguntas y respuestas (Wiley University
Services, 2023). Dividir el material en partes más pequeñas y fáciles de
comprender con retroalimentación rápida (pruebas) ayudó a los estudiantes a
mantenerse atentos y motivados durante largos períodos de distanciamiento social
(Søzmen et al., 2023).

Leong et al. (2021) revisaron la literatura sobre microaprendizaje entre 2006 y
2019. Los autores identificaron 476 publicaciones relevantes sobre el tema.
Comenzó como una metodología de transferencia de conocimientos basada en el
trabajo, el aprendizaje. En la última década, el concepto de microaprendizaje se
ha vuelto ampliamente conocido y famoso. El informe del Sistema Universitario
Global de Canadá (2022) señaló las deficiencias de la educación presencial
tradicional y cómo la educación flexible y basada en habilidades puede
abordarlas. En busca de relevancia, las instituciones encontraron que el
microaprendizaje era relevante para su adopción en la educación superior (Leong
et al., 2021).

Shatte y Teague (2020) investigaron la literatura sobre microaprendizaje basado
en tecnología en la educación superior. Los autores muestran la contribución del
enfoque de microaprendizaje al aumento de los resultados educativos. Los
estudiantes reportaron una mayor motivación y un mayor compromiso con el
material y los procesos de aprendizaje al estudiar porciones pequeñas y concisas
de material (Shatte y Teague). Garshasbi et al. (2021) analizaron el efecto del
microaprendizaje en la enseñanza de las disciplinas STEM (ciencia, tecnología,
ingeniería y matemáticas). En un mundo con abundante información, la capacidad
de los estudiantes para concentrarse en clases virtuales, ya sea en línea o
presenciales, es cuestionable. El microaprendizaje podría ser una solución
adecuada para abordar este desafío. En la educación en línea, una pequeña
porción de material, con un enfoque lógico y concluyente, preparada para su
entrega o consumo por parte de los estudiantes ha demostrado ser una forma
eficaz de transferencia de conocimiento (Garshasbi et al.).
\cite{article:elearning_future_trends_shahid}

\subsection{Idea}
Este artículo analiza el concepto de microaprendizaje en el contexto del
aprendizaje a lo largo del desarrollo profesional y presenta un caso de estudio
dentro de un proyecto de investigación. En este caso de estudio, desarrollamos
una solución web para la capacitación empresarial que incorpora el
microaprendizaje, optimizando el tiempo que los empleados dedican a su
formación. La solución busca ofrecer una experiencia de aprendizaje flexible y
adaptada al tiempo disponible de cada empleado. Además, las rutas de aprendizaje
se ajustan a los objetivos individuales, fomentando el desarrollo profesional
continuo.

\subsection{Objetivo}
El objetivo de este artículo es examinar cómo esta plataforma, mediante
actividades breves como videos, lecturas y cuestionarios, logra un aprendizaje
eficiente adaptado a las necesidades individuales de los empleados. Asimismo, se
explican los beneficios de la accesibilidad multiplataforma, que garantiza
continuidad en el proceso de formación, y los efectos positivos que este enfoque
ha demostrado tener en la eficiencia del aprendizaje y el crecimiento
profesional.

\subsection{Secciones}
En \textbf{Introducción}, se plantea la problemática, se define el objetivo y las metas de la investigación, se contextualiza con antecedentes y se presenta la idea central del trabajo.
En \textbf{Justificación}, se argumenta la relevancia del enfoque, respaldado por estudios previos.
En \textbf{Marco Teórico} se profundizan los conceptos clave y marca diferencias con otros enfoques educativos.
En \textbf{Trabajos Relacionados} se revisan herramientas similares y metodologías aplicadas en otros entornos empresariales.
En \textbf{Enfoque Propuesto} se describe la solución a ser desarrollada y la arquitectura de software a ser utilizada.
En \textbf{Diseño de la Investigación} se detalla la metodología utilizada para evaluar la efectividad del sistema.
En \textbf{Instrumentos} se presentan las herramientas utilizadas, como frameworks tecnológicos y plataformas de desarrollo.
En \textbf{Componentes} se listan los elementos clave del sistema a nivel software. Se vuelve a mencionar la arquitectura pero desde la implementación.
En \textbf{Procedimientos} se explica los pasos para la implementación de la solución.
En \textbf{Resultados} se analiza el impacto del sistema y se describen las pruebas realizadas.
En \textbf{Conclusión}, se evalúan los hallazgos obtenidos y se destacan las contribuciones del modelo de microaprendizaje.
En \textbf{Apéndices} se recopila información complementaria relevante.
En \textbf{Reconocimientos}, se agradecen contribuciones significativas al proyecto.
En \textbf{Referencias} se documentan fuentes y estudios clave que sustentan la investigación.
En \textbf{Biografía}, se presenta el perfil del autor.


\section{Justificación}
Los cambios tecnológicos, económicos y sociales actuales impulsan la necesidad
de nuevos conceptos y estrategias para apoyar el aprendizaje permanente. La
educación, incluida la formación en el trabajo, necesita transformaciones que
requieren renovación y formas innovadoras de adaptarse adecuadamente a nuestra
forma de vivir, trabajar y aprender en la actualidad.
\cite{article:microlearning_buchem}

Los términos que describen cómo trabajamos y aprendemos hoy en día, como ''trabajadores
del conocimiento'' (''knowledge workers''), ''nativos digitales'' (''Digital
Natives''), ''inmigrantes digitales'' ("Digital Immigrants"), ''estudiantes del
nuevo milenio'' (''new millennium learners'') y la ''Generación Net'' (''Net
Generation''), reflejan algunos cambios esenciales en las sociedades modernas,
''donde las tecnologías digitales forman parte inseparable de la vida cotidiana''
(''where digital technologies form an inextricable part of daily life''). El
estilo de vida de la nueva generación, es decir, de los nacidos a partir de la
década de 1980, pero también de las generaciones anteriores, se está viendo
fuertemente influenciado por Internet y las tecnologías relacionadas. Tanto los
estudiantes jóvenes como los mayores llevan sus portátiles a clases o reuniones
de trabajo, usan teléfonos móviles e internet para fomentar las redes sociales,
emplean dispositivos digitales para jugar y crear contenido, o realizan
múltiples tareas simultáneamente.
\cite{article:microlearning_buchem}

Estas nuevas tecnologías digitales que permiten la creación de contenido
generado por el usuario han dado lugar a una tendencia hacia los microformatos
(''microformats''), es decir, información breve, sencilla y específica. Junto con
los sistemas de publicación personal, como blogs o wikis, se ha vuelto bastante
fácil para cualquiera crear su propio contenido, incluido el microcontenido
(''microcontent''). El microcontenido, es decir, ''información publicada en formato
breve'', se relaciona más con un "enfoque formal de cómo presentar" el contenido
que con la calidad inherente del contenido en sí. Algunos ejemplos de
microcontenido son los podcasts, las entradas de blog, las páginas wiki o los
mensajes cortos en Facebook o X. La creación, publicación y compartición de
microcontenido en la web abre nuevas posibilidades para formas de aprendizaje
implícitas, informales e incidentales, como el microaprendizaje
(''microlearning''), término que se refiere a actividades breves de aprendizaje
con microcontenido.
\cite{article:microlearning_buchem}

\section{Marco Teórico}
% ¿Qué escribir en esta sección?
% Esta sección e introducción deben contener casi todas las referencias del paper.
% Acá tenemos que identificar los conceptos básicos de la investigación que serán empleados en las secciones sucesivas.
% En esta sección se orienta la investigación desde un punto de vista innovador marcando las posibles diferencias con otras investigaciones.

\subsection{Microaprendizaje}
El microaprendizaje se refiere a formas breves de aprendizaje y consiste en
actividades de aprendizaje breves, detalladas, interconectadas y poco acopladas,
con microcontenido (Lindner, 2006; Schmidt, 2007).
\cite{article:microlearning_buchem}

\subsection{Aprendizaje personalizado (AP)}
El aprendizaje personalizado (AP) es un entorno de aprendizaje digitalizado
diseñado para satisfacer los resultados de aprendizaje de un individuo y
adaptado a su conocimiento y experiencia.
\cite{article:elearning_future_trends_shahid}

\subsection{Aprendizaje adaptativo (AA)}
Término que describe una situación en la que los conocimientos impartidos se
adaptan a las necesidades de cada estudiante.
\cite{article:elearning_future_trends_shahid}

\section{Trabajos Relacionados}
% Los trabajos relacionados podrían ser herramientas que ya existen
% o que se mencionan en los paper que hacen parte de lo mencionado en el paper.

\section{Enfoque Propuesto}
Se desarrollará un Learning Management System (LMS) con un enfoque
centrado en el microaprendizaje, permitiendo una capacitación más
flexible y adaptativa para los usuarios.  

\subsection{Arquitectura}  
La solución adoptará una arquitectura cliente-servidor, con el
propósito de garantizar un funcionamiento eficiente y escalable. El diseño de la
arquitectura toma como referencia los enfoques planteados en diversos estudios
previos, incluyendo:  

\begin{itemize}  
    \item \textit{Design and Implementation of E-Learning Management System
    using Service Oriented Architecture} \cite{article:design_lms_soa_jabr}  
    \item \textit{An implementable architecture of an e-learning system}
    \cite{inproceedings:elearning_architecture_liu}  
    \item \textit{Microservice architecture in E-learning}
    \cite{article:microservice_architecture_elearning_milovanovic}  
\end{itemize}  

Estos artículos han influenciado el diseño de la solución, proporcionando los
conecptos y definiciones que permiten crear los modelos estructurales que
permiten mejorar la eficiencia del sistema y optimizar la distribución de los
contenidos de microaprendizaje dentro de un entorno LMS.  

\subsection{Innovación}  
La principal característica diferenciadora de esta solución respecto a otras
disponibles en el mercado es su integración con plataformas de mensajería. Este
enfoque busca facilitar la notificación automática a los usuarios sobre nuevos
cursos disponibles, fomentando un ritmo de aprendizaje continuo y adaptable a
sus necesidades.  

Inicialmente, la integración se realizará con la plataforma Telegram,
permitiendo el envío de actualizaciones en tiempo real. No obstante, se
contempla la posibilidad de extender la compatibilidad a otras plataformas de
mensajería en futuras investigaciones, según la demanda y los requerimientos del
sistema.  

La decisión de implementar esta integración se fundamenta en los resultados
obtenidos en diversas investigaciones, tales como:  

\begin{itemize}  
    \item \textit{The Usability Of WhatsApp Messenger as Online
    Teaching-learning Media} \cite{article:whatsapp_online_teaching_darman}  
    \item \textit{Using WhatsApp to Support Communication in Teaching and
    Learning} \cite{inproceedings:whatsapp_support_teaching_ujakpa}  
    \item \textit{Perception of the use of LMS/i-Learn portal and Telegram}
    \cite{article:perceptions_lms_telegram_hassim}  
    \item \textit{Assessment of the Applicability of using Telegram as a
    Learning Management System} \cite{article:telegram_lms_elebaed}  
\end{itemize}  

Según el artículo \textit{Assessment of the Applicability of using Telegram as a
Learning Management System}:  

\begin{quote}  
Los resultados analizados de los experimentos diseñados concluyen que Telegram
es un medio aceptable para comunicar los materiales de aprendizaje (vídeos y
apuntes) y proporciona un nivel aceptable de interactividad en el aprendizaje.
Además, la tasa media de éxito de los estudiantes en los cursos impartidos a
través de Telegram fue mejor que la de los cursos impartidos en el aula.
\end{quote}  
\cite{article:telegram_lms_elebaed}

Con base en estos hallazgos, se ha decidido iniciar la integración con
Telegram como plataforma principal, asegurando una comunicación
efectiva entre el LMS y los usuarios, y mejorando la accesibilidad a los
contenidos educativos.  


\section{Diseño de la investigación}
Esta sección describe la metodología utilizada para evaluar la
factibilidad del diseño de un Learning Management System (LMS) basado
en microaprendizaje con integración a plataformas de mensajería. Se analiza la
viabilidad técnica y estructural del sistema, considerando aspectos clave como
la arquitectura, la escalabilidad y la integración con herramientas externas.  

\subsection{Metodología}  
Para determinar la factibilidad del diseño, se lleva a cabo un análisis
exploratorio basado en pruebas de concepto y revisión de modelos previos. La
metodología empleada abarca los siguientes aspectos:  

\begin{itemize}  
    \item \textbf{Evaluación tecnológica}: Se examinan las tecnologías
    existentes que permiten la implementación de un LMS con soporte para
    microaprendizaje, identificando su compatibilidad con arquitecturas
    cliente-servidor y la integración con plataformas externas.  
    \item \textbf{Análisis arquitectónico}: Se diseña un modelo conceptual de la
    arquitectura del sistema, fundamentado en enfoques propuestos en trabajos
    previos.  
    \item \textbf{Pruebas de integración}: Se exploran mecanismos de
    comunicación entre el LMS y plataformas de mensajería como Telegram,
    verificando su viabilidad dentro del diseño.  
    \item \textbf{Escalabilidad y adaptabilidad}: Se evalua la capacidad del
    sistema para ajustarse a distintos escenarios empresariales y soportar una
    expansión futura.  
\end{itemize}  

\subsection{Procedimiento}  
El análisis de factibilidad se lleva a cabo mediante las siguientes etapas:  

\begin{enumerate}  
    \item \textbf{Revisión de literatura}: Se investigan sistemas LMS con
    características similares, identificando fortalezas y limitaciones.  
    \item \textbf{Diseño conceptual}: Se definen los módulos clave y su
    interacción dentro de la arquitectura propuesta.  
    \item \textbf{Evaluación de integración}: Se prueban mecanismos para la
    comunicación entre el LMS y Telegram, verificando la posibilidad de
    automatizar notificaciones.  
    \item \textbf{Análisis de viabilidad}: Se identifican potenciales desafíos
    técnicos y soluciones para asegurar la implementación del sistema en un
    entorno empresarial.  
\end{enumerate}  

Este enfoque metodológico pertmite obtener un marco claro sobre la viabilidad
del sistema propuesto, proporcionando las bases necesarias para futuras fases de
desarrollo e implementación.  


\section{Instrumentos}
\blindtext

% La sección de Instrumentos contiene: github, editor de texto, herramientas utilizadas, frameworks,…

\section{Componentes}
% Cuando se desarrolla una idea de un prototipo, se necesita una sección donde listar los distintos componentes hardware o software que se utilizará en el proceso de investigación.
% En los componentes hardware normalmente se encuentran los sensores. Los componentes software están más orientados a la arquitectura y
% al diseño de las aplicaciones, y no a los framework y lenguaje utilizados.

% Spans both columns, making it useful for large images or diagrams.
\begin{figure}[htbp]
    \centering
    \includegraphics[width=\linewidth]{diagrama_arquitectura.png}
    \caption{Diagrama de Arquitectura}
    \label{fig:diagrama_arquitectura}
\end{figure}

\begin{figure}[htbp]
    \centering
    \includegraphics[width=\linewidth]{diagrama_arquitectura_alto_nivel.png}
    \caption{Diagrama de Arquitectura - Alto Nivel}
    \label{fig:diagrama_arquitectura_alto_nivel}
\end{figure}

\begin{figure}[htbp]
    \centering
    \includegraphics[width=\linewidth]{diagrama_arquitectura_cliente.png}
    \caption{Diagrama de Arquitectura - Cliente}
    \label{fig:diagrama_arquitectura_cliente}
\end{figure}

\begin{figure}[htbp]
    \centering
    \includegraphics[width=\linewidth]{diagrama_arquitectura_servidor.png}
    \caption{Diagrama de Arquitectura - Servidor}
    \label{fig:diagrama_arquitectura_servidor}
\end{figure}

\begin{figure}[htbp]
    \centering
    \includegraphics[width=\linewidth]{diagrama_arquitectura_servicios.png}
    \caption{Diagrama de Arquitectura - Servicios}
    \label{fig:diagrama_arquitectura_servicios}
\end{figure}

\begin{figure}[htbp]
    \centering
    \includegraphics[width=\linewidth]{diagrama_arquitectura_integraciones.png}
    \caption{Diagrama de Arquitectura - Integraciones}
    \label{fig:diagrama_arquitectura_integraciones}
\end{figure}

\begin{figure}[htbp]
    \centering
    \includegraphics[width=\linewidth]{diagrama_arquitectura_servidor_bot.png}
    \caption{Diagrama de Arquitectura - Servidor Bot}
    \label{fig:diagrama_arquitectura_servidor_bot}
\end{figure}

\begin{figure}[htbp]
    \centering
    \includegraphics[width=\linewidth]{diagrama_arquitectura_bot.png}
    \caption{Diagrama de Arquitectura - Bot}
    \label{fig:diagrama_arquitectura_bot}
\end{figure}




\section{Procedimientos}
Esta sección describe los pasos seguidos para la implementación de la solución
LMS basada en microaprendizaje, asegurando un desarrollo estructurado y
eficiente.  

\subsection{Configuración de la Arquitectura}  
Para la implementación del sistema, se define una arquitectura cliente-servidor
compuesta por varios módulos clave. Se diseñan los componentes principales
para garantizar una interacción fluida entre los usuarios y el sistema,
asegurando accesibilidad y escalabilidad.  

\subsection{Implementación del Cliente}  
La interfaz de usuario es desarrollada utilizando React, optimizada
con Vite para mejorar el rendimiento. Los pasos de implementación
incluyen:  

\begin{itemize}  
    \item Configuración del entorno de desarrollo con React.  
    \item Desarrollo de componentes dinámicos para la presentación de cursos.  
    \item Integración con API REST para la comunicación con el backend.  
    \item Gestión del estado y almacenamiento local de preferencias del usuario.  
\end{itemize}  

\subsection{Desarrollo del Servidor}  
El backend, construido en Node.js con Express, se encarga de gestionar
la lógica de negocio y la seguridad del sistema. Su implementación se realiza
siguiendo los siguientes pasos:  

\begin{itemize}  
    \item Creación de endpoints para la gestión de usuarios, cursos e
    integraciones.  
    \item Implementación de autenticación y autorización mediante JWT.  
    \item Integración con MongoDB para el almacenamiento de datos estructurados.  
    \item Comunicación con servicios externos provistos a través de contenedores
    Docker para garantizar escalabilidad.  
\end{itemize}  

\subsection{Integración con Telegram}  
Para mejorar la interacción con los usuarios, se desarrolla un bot de Telegram
que permite la notificación automática de nuevos cursos. La integración sigue
estos pasos:  

\begin{itemize}  
    \item Creación del bot utilizando la API de Telegram.  
    \item Desarrollo de un servidor intermedio en Flask para gestionar las
    solicitudes.  
    \item Configuración de comunicación entre el bot y el sistema LMS.  
\end{itemize}  

\subsection{Despliegue y Mantenimiento}  
Lo servicios, el bot server y el bot se configuran para ejecutarse en
contenedores Docker, asegurando portabilidad y facilidad de mantenimiento.
El servidor, por su parte, se despliega a un ambiente configurado para su
correcta ejecución.
Para el despliegue se llevan a cabo las siguientes acciones:  

\begin{itemize}  
    \item Creación de archivos Docker para cada módulo del sistema que haga uso del mismo.  
    \item Automatización de despliegues utilizando Docker Compose.  
\end{itemize}  

Este enfoque metodológico permite una implementación estructurada del sistema,
facilitando la optimización de sus funcionalidades y asegurando la integración
efectiva de microaprendizaje en un entorno empresarial.  


\section{Resultados}
% Sección obligatoria.
% ¿Qué escribir en esta sección?
% Se hace un análisis de los resultados obtenidos. Esta sección contiene detalles acerca de cómo y por qué se obtuvieron dichos resultados. 
% Los resultados necesitan una descripción detallada de cómo fueron obtenidos (cuántos intentos, cuántas pruebas y cuántos casos exitosos).

El desarrollo de la solución web de capacitación empresarial basada en
microaprendizaje ha demostrado ser factible, cumpliendo con los objetivos
planteados en el diseño inicial. La plataforma implementa un sistema flexible
que facilita el acceso al aprendizaje y optimiza el tiempo de formación de los
empleados.

Como resultado, se han implementado módulos clave que permiten personalizar el
contenido educativo, garantizar accesibilidad desde múltiples dispositivos y
facilitar la interacción mediante plataformas externas. Uno de los aspectos
distintivos del sistema es la integración con Telegram, que mejora la
automatización de notificaciones para los usuarios.

Integrar el sistema de capacitación
con Telegram ha permitido ir optimizando la interacción y automatización de notificaciones para
los usuarios. A continuación, se presentan las principales pantallas que
demuestran esta funcionalidad.

\begin{figure}[htbp]
    \centering
    \includegraphics[width=\linewidth]{placeholder_interfaz_telegram.png}
    \caption{Interfaz del sistema con configuración de integración a Telegram}
    \label{fig:interfaz_integracion_telegram}
\end{figure}

La Figura \ref{fig:interfaz_integracion_telegram} muestra la sección del sistema
donde se encuentra la configuración para integrar Telegram. Desde esta interfaz,
los usuarios pueden activar la funcionalidad y gestionar sus preferencias de
notificación.

\begin{figure}[htbp]
    \centering
    \includegraphics[width=\linewidth]{placeholder_notificacion_telegram.png}
    \caption{Notificación generada por el sistema en Telegram}
    \label{fig:notificacion_telegram}
\end{figure}

La Figura \ref{fig:notificacion_telegram} presenta una notificación enviada al
usuario a través de Telegram. Estas notificaciones permiten alertar sobre nuevos
cursos disponibles y recordatorios de formación, asegurando una comunicación
eficiente dentro del proceso de aprendizaje.

\begin{figure}[htbp]
    \centering
    \includegraphics[width=\linewidth]{placeholder_interfaz_curso.png}
    \caption{Interfaz del sistema tras acceder desde la notificación}
    \label{fig:interfaz_curso_ejemplo}
\end{figure}

La Figura \ref{fig:interfaz_curso_ejemplo} muestra la vista de un curso al que
el usuario accede tras hacer clic en la notificación recibida en Telegram. Desde
esta pantalla, el usuario puede explorar los detalles del curso y realizar su
inscripción de manera rápida, garantizando una transición fluida entre la
plataforma de mensajería y el sistema de capacitación.

Los principales logros de la solución incluyen:
\begin{itemize}
    \item Personalización del contenido educativo según el perfil y objetivos
    individuales de los empleados.
    \item Integración con plataformas de mensajería para optimizar la
    interacción y accesibilidad.
    \item Accesibilidad multiplataforma, permitiendo el uso desde dispositivos
    móviles y de escritorio.
\end{itemize}

Esto demuestra por un lado la viabilidad del sistema y su capacidad de
adaptación a diversos entornos de capacitación empresarial. Por otro lado, estos
resultados reflejan la integración efectiva del sistema con plataformas
externas, mejorando la accesibilidad y optimizando la experiencia del usuario en
el proceso de aprendizaje.


\section{Conclusión}
% Sección obligatoria.
% ¿Qué escribir en esta sección?
% Reafirmar la idea en este apartado. Una versión simple.
% Solamente informamos si los resultados fueron positivos o negativos. Si resultó exitosa o no.
% A su vez en la conclusión también es necesario explicar los obstáculos del paper e indicar los posibles trabajos futuros.

% ¿Cómo debe escribirse esta sección?
% No tiene extensión definida, puede ser tan extensa como se requiera

Este trabajo demuestra que la solución web de capacitación empresarial basada en
microaprendizaje personalizado es una estrategia efectiva para mejorar el
aprendizaje y el desarrollo profesional. Adaptando el contenido a las
necesidades individuales y facilitando el acceso desde cualquier dispositivo, se
ha logrado optimizar el tiempo de los empleados, garantizando un proceso de
formación continuo y con impacto significativo en el ámbito laboral.

En un entorno laboral dinámico y en constante evolución, es fundamental que las
competencias clave se adquieran, actualicen y mantengan de manera permanente.
Los métodos tradicionales de formación suelen ser rígidos y poco eficientes para
responder a las necesidades cambiantes del mundo profesional. En contraste, el
microaprendizaje, integrado en plataformas digitales como la solución web
presentada, permite una capacitación ágil y accesible. A través de contenidos
breves y personalizados, combinados con accesibilidad multiplataforma, esta
metodología facilita el aprendizaje progresivo, adaptado a los ritmos de trabajo
y promoviendo el desarrollo profesional continuo.

\appendices
% Este apéndice proporciona información complementaria para reforzar la solución
% propuesta sin sobrecargar el cuerpo principal del paper.
% - Especificaciones de software y hardware requeridas.
% - Diagramas de arquitectura del sistema LMS.
% - Algoritmos utilizados para personalizar el microaprendizaje.
% - Capturas de pantalla o ejemplos de la interfaz del LMS.
% - Resultados detallados de pruebas piloto o estudios de usabilidad.
% - Comparación con otros LMS tradicionales y sus limitaciones.
% - Formularios utilizados para evaluar la efectividad del sistema.
% - Respuestas clave de usuarios o docentes sobre la experiencia de aprendizaje.
% - Segmentos relevantes del código para la implementación del LMS.
% - Scripts utilizados para la automatización de contenido microlearning.
% - Fórmulas aplicadas en la recomendación de contenidos.
% - Estadísticas detalladas sobre mejora en el aprendizaje.
% - Glosario de términos técnicos utilizados.
% - Documentación técnica adicional.

% La opción [H] en \begin{figure}[H] proviene del paquete float y significa ''Colocar la figura exactamente aquí''.

\appendices
\section{Interfaz del Sistema en Imágenes}

\begin{figure}[H]
    \centering
    \includegraphics[width=\linewidth]{plataforma/plataforma_01.png}
    \caption{Interfaz del sistema tras acceder desde la notificación}
    \label{fig:interfaz_sistema_notifiacion}
\end{figure}

\section{Lecciones en Imágenes}

\section{Integración con Telegram en Imágenes}

\begin{figure}[H]
    \centering
    \includegraphics[width=\linewidth]{integracion_telegram/integracion_telegram_01.png}
    \caption{Integración deshabilitada}
    \label{fig:integracion_telegram_deshabilitada}
\end{figure}

\begin{figure}[H]
    \centering
    \includegraphics[width=\linewidth]{integracion_telegram/integracion_telegram_03.png}
    \caption{Registro de usuario en el bot}
    \label{fig:integracion_telegram_registro_bot}
\end{figure}

\begin{figure}[H]
    \centering
    \includegraphics[width=\linewidth]{integracion_telegram/integracion_telegram_04.png}
    \caption{Integración habilitada}
    \label{fig:integracion_telegram_habilitada}
\end{figure}

\section{Integración con Sistema de Correos Electrónicos en Imágenes}

\begin{figure}[H]
    \centering
    \includegraphics[width=\linewidth]{mail/mail_01.png}
    \caption{Correo de OTP}
    \label{fig:correo_electronico_otp}
\end{figure}
\begin{figure}[H]
    \centering
    \includegraphics[width=\linewidth]{mail/mail_02.png}
    \caption{Correo de confirmación de registro a curso}
    \label{fig:correo_electronico_confirmacion_registro_curso}
\end{figure}


\section*{Reconocimientos}
% \begin{comment}
Extiendo mi gratitud a la Universidad de Palermo (UP), a sus docentes y autoridades, por proporcionarme
las herramientas necesarias para la realización de este trabajo, contribuyendo significativamente a mi formación académica.

Quiero hacer una mención especial al Magíster Javier Rabuch, Director de las carreras de Inteligencia Artificial,
por su valiosa orientación y apoyo durante el desarrollo de esta investigación, tanto en su rol de profesor como
en su función de tutor. Su guía ha sido clave en el proceso de este estudio.
% \end{comment}

% Es necesario diferenciar las referencias entre la justificación de la idea,
% la problemática, las investigaciones relacionadas que ya existen para un problema similar
% y el conocimiento técnico (Marco Teórico) para que podamos construir la solución

\bibliographystyle{IEEEtran}
\bibliography{bib/IEEEabrv,bib/references}
% Permite mostrar todas las referencias, incluso aquellas que no fueron citadas.
\nocite{*}
% Ajusta el espacio posterior a las referencias.
% \vspace{-425pt}



\begin{IEEEbiography}[
        {\includegraphics[width=1in,height=1.25in,clip,keepaspectratio]{images/author.jpg}}
    ]
    {Andrés Isaac Biso} es estudiante avanzado de la carrera ''Licenciatura en
    Informática'' en la Universidad de Palermo (UP).
    Es Analista Universitario en Sistemas, título otorgado por la misma institución.
    Actualmente trabaja en GlobalLogic, donde se desempeña como desarrollador de software.
    Sus áreas de interés en investigación incluyen la integración de tecnología y educación.
\end{IEEEbiography}


\end{document}
